\documentclass[12pt]{article}
\usepackage[T2A]{fontenc}
\usepackage[UTF8]{inputenc}
\usepackage[english,russian]{babel}
\usepackage[left=3cm,right=1.5cm,top=2cm,bottom=2cm]{geometry}
\usepackage{mathtools}

\title{Task2}
\author{Цуман Александр}
\date{11.07.2014}

\begin{document}

\begin{center}\bf MathCad. \rm\end{center}

1. Ввести матрицу \it A \rm --- размерности $3 \times 4$, где $a_{ij} = \sin (0.3i+2j)$. Посчитать выражение
\[ \it A \rm + 
\begin{pmatrix} 
	& 5 & \\ 
	& 4 & \\ 
	& 6 & \\ 
	& 4 & 
\end{pmatrix}
	\ast 
\begin{pmatrix}
	& 7 & -2 & 0.3 &
\end{pmatrix}
	-
\begin{pmatrix}
	& 1 & 0 & 2 &\\
	& -3 & 2 & 6 &\\
	& 0 & 0 & 3 &\\
\end{pmatrix}
	.
\]
Высчитать определитель данного выражения.

2. На одном рисунке построить графики функций:
\begin {equation*}
	f_1(x)=e^{sin(\frac{x}{2})},
	f_2(x)=cos(x)\cdot sin(x)+2, x\in[3,6]
\end {equation*}

3. Построить график функции
\begin {equation*}
	e^{sin(x) \ast cos(y)}
\end {equation*}


4. Написать программный модуль, для вычисления функции \it f(x,y) \rm - разложение в ряд 
	\\Тейлора функции $ \sqrt{x} $, в окрестностях точки y. Сделать проверку на некоректные входные 
	\\данные.	

5. Вычислить
\begin {equation*}
	\int \frac{(t-3)\sqrt{t+1}}{t^2}dt;
\end {equation*}
\begin {equation*}
	\frac{d}{dm}\biggl[e^{m^2} \ast \tg(\frac{3}{m})\biggr];
\end {equation*}
\begin {equation*}
	a=\frac{\sqrt{5 \ast \sin(34)}}{|\sin(28 \ast 0.4 - 15\cos(3))|};
\end {equation*}
\begin {equation*}
	b=\tg(4) \ast \int\limits_{-1}^2 \sin(t^3)e^{4+t}dt;
\end {equation*}
\begin {equation*}
	a+b;
\end {equation*}

6. Раскрыть скобки в выражении\\ 	
\begin {equation*}
	{(x_3+3)}^2+{(x_2-x_3)^3};
\end {equation*}
Разложить в ряд $$e^x$$

\begin{center}\bf MatLab. \rm\end{center}

7. Ввести матрицы\\
\begin {center}
	$ A =
	\begin{pmatrix} 
		& 8 & 5 & 1 &\\
		& -3 & -5 & 0 &\\
		& 56 & 12.67 & 3.09 &\\
	\end{pmatrix}
	$
	и
	$ B =
	\begin{pmatrix} 
		& -4 & 0.4 & 21.8 &\\
		& -0 & 67.02 & 12.96 &\\
		& 4 & 45.28 & -3 &\\
	\end{pmatrix}
	$
\end {center}

8. В матрице возвести в квадрат все неотрицательные элементы, отрицательные поделить
	\\на определитель A (реализовать при помощи управляющих конструкций).

9. Построить график функции $ \sqrt{\frac{x+3}{x-2}} $ на промежутке [3,5].

\end{document}